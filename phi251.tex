%Include things
\documentclass{report}
\usepackage{graphicx}
\usepackage{amssymb}
\usepackage[left=0.5in,top=1.0in,right=0.5in,bottom=1.0in]{geometry} % Document margins
\usepackage{amsmath}
 \usepackage[noindentafter]{titlesec}
\titleformat{\chapter}
  {\normalfont\huge\bfseries}{\noindent Chapter \thechapter:}{0.5em}{}
% this alters "before" spacing (the second length argument) to 0
\titlespacing*{\chapter}{0pt}{0pt}{0pt}
%My own macros


\titleformat{\section}
  {\normalfont\Large\bfseries}{\noindent Lesson \thesection:}{0.5em}{}

\title{PHI251 - Logic}
\author{Grant Griffiths}
\date{Spring 2012}
\begin{document}
   \maketitle
   \tableofcontents


\chapter{Fundamental Concepts and Truth-Functional Analysis}
	\section{Introduction}
		\begin{itemize}\addtolength{\leftskip}{2em}
			\item \textbf{Argument}: When reasons are given to justify a belief
			\item\textbf{ Argument Indicator Words}: indicate that an conclusion is being presented
			\begin{itemize}\addtolength{\leftskip}{4em}
				\item So 
				\item Hence
				\item Thus 
				\item Therefore
				\item It must be that
			\end{itemize}
			\item\textbf{Premise Indicator Words}: indicate that a premise is being presented
			\begin{itemize}\addtolength{\leftskip}{4em}
				\item For 
				\item Since
				\item Because
				\item Due to the fact that
			\end{itemize}
			\item \textbf{Validity}
			\begin{itemize}\addtolength{\leftskip}{4em}
				\item In a valid argument, if the premises were all true,  the conclusion would also be true
				\item In a valid argument, it is not possible for the conclusion to be false when all the premises are true
			\end{itemize}
			\item \textbf{Invalidity}
			\begin{itemize}\addtolength{\leftskip}{4em}
				\item When the premises are true and the conclusion is false. 
				\item When an argument's premises do not truly prove it's conclusion.
			\end{itemize}
			\item \textbf{Soundness}: A sound argument is an argument with both of the following features:
			\begin{itemize}\addtolength{\leftskip}{4em}
				\item It is valid
				\item All of its premises are true
			\end{itemize}
			\item \textbf{Validity and Consistency}
			\begin{itemize}\addtolength{\leftskip}{4em}
				\item Any argument is valid if and only if it would be inconsistent to assert all of it's premises but deny it's conclusion.
			\end{itemize}
			\item \textbf{Considering the possibilities}
			\begin{itemize}\addtolength{\leftskip}{4em}
				\item To establish that an argument is valid , it seems that we must somehow show that no possible situation exists in which the premises are true with the conclusion false.
				\item Need just one possible situation to show invalidity
			\end{itemize}
		\end{itemize}
		
	\section{Truth-Functional Representation}
	\begin{itemize}\addtolength{\leftskip}{2em}
	
		\item \textbf{Conjunction}
		\begin{itemize}\addtolength{\leftskip}{4em}
			\item First conjunct, Conjunct, Second Conjunct
			\item $A\wedge B$ = AND
		\end{itemize}
		
		\item \textbf{Disjunction}
		\begin{itemize}\addtolength{\leftskip}{4em}
			\item First Disjunct, Disjunct, Second Disjunct
			\item $A\vee B$ = OR
		\end{itemize}
		
		\item\textbf{Formulas}
		\begin{itemize}\addtolength{\leftskip}{4em}
			\item We use formulas to represent sentences
			\item if P is any formula, then so is ~P
			\item if P and Q are formulas, then $P\wedge Q$ is a formula
			\item if P and Q are any formulas, then $P\vee Q$ is a formula
		\end{itemize}
			
		\item\textbf{Advantages of Formulas}
		\begin{itemize}\addtolength{\leftskip}{4em}
			\item Formulas can be very complex
			\item They can express complex claims clearly and compactly
			\item English sentences can be long
		\end{itemize}
	
		\item\textbf{Major connective}
		\begin{itemize}\addtolength{\leftskip}{4em}
			\item The logical connector used last in the process of constructing the formula out of its parts
		\end{itemize}
		
		
		\item\textbf{Sentence Forms}
		\begin{itemize}\addtolength{\leftskip}{4em}
			\item Patterns or structural frameworks, for sentences
			\item Look like formulas
		\end{itemize}
	\end{itemize}
			
	\section{Truth-Functional Analysis}
		\addtolength{\leftskip}{4em}
		\bgroup
		\def\arraystretch{1.5}%
		\small\begin{tabular}{ | l | p{7cm} | p{7cm} |}
			
			\hline
			  & \textbf{Shown by Considering Every Possible Case }& \textbf{Shown by Example}\\
			\hline
			\textbf{Argument} &
			\textbf{Valid}: In every possible case, if premises are true, conclusion is also true. &
			\textbf{Invalid}: In at least one possible case, premises are true, conclusion is false.\\
			
			\hline
			\textbf{Set of Sentences} &
			\textbf{Inconsistent}: In every possible case, at least one is false. & 
			\textbf{Consistent}: In at least one possible case, all are true\\
			
			\hline
			\textbf{Pair of Sentences} & 
			\textbf{Equivalent}: In every possible case, they have the same truth-value. & \textbf{Not equivalent}: In at least one possible case, they have different truth-values. \\
			
			\hline
			\textbf{Sentence} &
			\textbf{Tautologous}: In every possible case, true \newline
			\textbf{Contradictory}: In every possible case, false. & 
			\textbf{Contingent}: True in at least one possible case and false in at least one possible case\\
			\hline
		\end{tabular}
		\egroup
	\section{Conditionals}
		\begin{itemize}\addtolength{\leftskip}{4em}
			\item False ONLY when the antecedent is true and the consequent is false.
			\item For this outline, I've represented false as 0 and true as 1.
		\end{itemize}
		
	\section{Truth Tables}
	\begin{itemize}\addtolength{\leftskip}{2em}
	\item Conjunction (AND)\newline\newline
 			\bgroup
			\def\arraystretch{1.5}%
			\small\begin{tabular}{ | l | l | p{1.25cm} |}
				
				\hline
				\textbf{P}& 
				\textbf{Q}&
				$\mathbf{P\wedge Q}$\\
				\hline
	        	
	        	0& 
				0&
				0\\
				\hline
				
	        	0& 
				1&
				0\\
				\hline
				
	        	1& 
				0&
				0\\
				\hline
				
	        	1& 
				1&
				1\\
				\hline
			\end{tabular}
			\egroup
	\item Disjunction (OR)\newline\newline
 			\bgroup
			\def\arraystretch{1.5}%
			\small\begin{tabular}{ | l | l | p{1.25cm} |}
				
				\hline
				\textbf{P}& 
				\textbf{Q}&
				$\mathbf{P\vee Q}$\\
				\hline
	        	
	        	0& 
				0&
				0\\
				\hline
				
	        	0& 
				1&
				1\\
				\hline
				
	        	1& 
				0&
				1\\
				\hline
				
	        	1& 
				1&
				1\\
				\hline
			\end{tabular}
			\egroup
		\item Conditional (If/then)\newline\newline
	 			\bgroup
				\def\arraystretch{1.5}%
				\small\begin{tabular}{ | l | l | p{1.25cm} |}
					
					\hline
					\textbf{P}& 
					\textbf{Q}&
					$\mathbf{P\Rightarrow Q}$\\
					\hline
		        	
		        	0& 
					0&
					1\\
					\hline
					
		        	0& 
					1&
					1\\
					\hline
					
		        	1& 
					0&
					0\\
					\hline
					
		        	1& 
					1&
					1\\
					\hline
				\end{tabular}
				\egroup
	\end{itemize}
\chapter{Derivations and Models}
	\section{Inference Rules}
	\section{Equivalence Rules}
	\section{Conditional Proof and Indirect Proof}
\chapter{Quantificational Logic}
	\section{The Language}
	\section{Derivations}
	\section{Predicate Logic Truth-Trees}
	\section{Unrestricted Quantification}
	\section{English Arguments}
	\section{Identity, Definite Descriptions, and Function Symbols}
	\section{Quantifier Semantics}

\newpage

\begin{center}
\textbf{\huge{Bibliography}}
\end{center}
\textbf{Book used:} Modern Formal Logic (Second Edition)\newline
\textbf{Professor:} Notes from Dr. Mark Brown's Spring 2012 course

\end{document}